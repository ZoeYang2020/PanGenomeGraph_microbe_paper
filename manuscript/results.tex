For requirements for a specific article type please refer to the Article Types on any Frontiers journal page. Please also refer to  \href{http://home.frontiersin.org/about/author-guidelines#Sections}{Author Guidelines} for further information on how to organize your manuscript in the required sections or their equivalents for your field
\subsection{Overview of the pangenome graph workflow}
A pangenome is defined as the comprehensive collection of whole-genome sequences from multiple individuals within a clade[39]. This collective genomic dataset can be further divided into two distinct components: the core genome, which includes genes present in all individuals, and the accessory genome, consisting of genes found only in a subset of individuals[40](Fig.1a). Pangenome graphs are pangenomes stored in graph models that can capture the entire genetic variation among genomes in a population or of a set of related organisms[15, 16, 19](Fig.1b). 
In this study, we have developed a pangenome graph pipeline for microbial genomics, consisting of graph construction using the PanGenome Graph Builder (PGGB)[30], graph manipulation through the Optimized Dynamic Genome/Graph Implementation (odgi)[41], and calling for Next-Generation Sequencing (NGS) data using the vg toolkit[16] (Fig.1c). The PGGB pipeline, a reference-free method, constructs pangenome graphs by employing all-to-all whole genome alignments with wfmash, graph induction via seqwish, and progressive normalization using smoothxg and gfaffix[30]. Odgi facilitates graph manipulation tasks such as visualization, and extraction of distances among paths in the graph, enabling phylogenetic analysis[41]. By utilizing the pangenome graph created with PGGB, it is possible to simultaneously identify various genetic variations, including structural variants, rearrangements, and small variants such as single nucleotide polymorphisms and insertions/deletions, through vg deconstruction. Additionally, we utilized the vg toolkit for analyzing NGS data against the graph for read mapping and variant calling[16]. 
The pangenome graph construction by pggb was demonstrated to be effective across various datasets, though the resulting graphs varied significantly based on the complexity of the input genomes (Table S3). The total run times for PGGB were 10.8 minutes, 8.3 minutes, and 4392 minutes, and the maximum memory usage was 1.96 GB, 2.11 GB, and 40.52 GB for 4Sim, 3ST, and 130NM, respectively. In the case of the 130NM genomes, employing the PGGB tool with the "-x auto" option enabled for the giant component heuristic resulted in a total execution time of 2787 minutes and a peak memory usage of 22.98 GB. Notably, the generated graph remained identical to the one obtained without this option. 
\subsection{High Consistency between Variations in the 4Sim Genome Graph and Ground Truth}
The final smooth graph for the 4Sim genomes spanned 2,260,981 bp and consisted of 30,033 nodes and 40,273 edges. Mauve alignment (Fig.2A) supported our observations, as inversions were displayed as bubbles in the 2D visualization (Fig.2B) and as inverted sequences in the 1D visualization (Fig.2C). The VCF file indicated that inversions were identified as different genotypes across various genomes. It is important to note that some variations in the graph did not correspond to the ground truth due to indel region alignment discrepancies (Fig.1D). Upon manual inspection of these sites, we found that they represented the same variation but were aligned to either the left or the right of the indels in the graph compared to the ground truth. We detected four, three, and two false negative small variations for ST154Sim, ST41Sim, and ST42Sim, respectively, in comparison to ST42. Additionally, we identified seven false positive small variants in ST154Sim. Despite the relatively simple nature of the simulated genomes, the agreement between the variations in the graph and the ground truth implies that the pangenome graph generated by PGGB is reliable (Table S4).
\subsection{Enhanced Mapping Rates and Comparable Variant Calling in Graph-Based Analysis of Simulated NGS Data}
Utilizing a pangenome graph reference for the analysis of NGS data can significantly enhance mapping rates. We conducted an evaluation by comparing the mapping rates of simulated NGS data based on the 4Sim graph to each of the linear references: ST42, ST42Sim, ST41Sim, and ST154Sim. All datasets were mapped to the graph, yielding a 100% mapping rate. Although the rates of NGS data aligned to each single linear reference were all over 0.99, a bias was observed in the linear reference mapping rates (Fig.3A). Our findings indicate that the use of a pangenome graph reference can greatly improve mapping rates in NGS data analysis.
The pangenome graph integrates various genomic variants, making it possible to genotype variants in NGS datasets. Interestingly, the genotyped results demonstrated high consistency across the eight simulated NGS datasets (Fig.3B and Table S6). The ST42Sim group exhibited the highest number of variants, consistent with the original simulation of 5000 SNPs and 200 indels. Moreover, the ST41Sim group displayed more identified variants compared to ST41 and ST42Mut, while the ST154Sim group revealed more variants compared to ST154 and ST154Sim.
Variant calling for NGS data against the graph using the vg toolkit differs slightly from single linear reference-based calling. In the absence of a defined path for variant calling, the process will call variants against the paths in alphabetical order (e.g., core genome part from A path, accessory genomes from B path, etc.). The variant call format (VCF) file includes a PASS column to mark variants that pass all filters, and the genotype (GT) describes the identified genotype in each sample. Since we employed haplotype bacterial genomes, variants with PASS but GT not equal to 1|1 were classified as errors, while those with PASS and GT=1|1 were classified as high-confidence variants. For each simulated NGS group, high-confidence variants exhibited consistency. Interestingly, the ST41 and ST154 groups exhibited the lowest percentage of high-confidence variants, which may be attributed to their greater genomic diversity and the absence of a reference from both groups in the graph. Including one reference from these groups in the pangenome graph led to an improvement in the percentage of high-confidence variants (Table S8 and Fig.3C). Furthermore, as NC_017518 of ST42 was the first path from the graph for variant calling, the ratio of high-confidence variants to the number of simulated variants was higher in ST42 (0.944 to 0.959) and ST42Sim (0.959 to 0.9706), but relatively lower in ST154Mut (0.8755 to 0.9000) and ST154Sim (0.8792 to 0.9049) (Fig.3D).
\subsection{Enhanced Mapping of NZmenB Real NGS Data to Pangenome Graph}
The three sequence types (STs) represent the three primary clades of NZmen (Fig. 4A). The final refined graph for the 3STs spanned 2,304,073 bp, consisting of 23,323 nodes and 31,325 edges. The inverted regions are consistent in both the Mauve alignment (Fig. 4B) and the 1D graph visualization (Fig. 2C).
We mapped each group of genomes (ST154, ST41, and ST42) to the respective linear references - ST154, ST41, ST42, and the 3STs graph. Despite the higher diversity of the compared genomes, particularly within the ST41 group, the mapping rate is higher when mapped to the graph as opposed to a single linear reference (Fig. 4D, Table S10). For instance, the reads of ST154 mapped to ST154, ST41, ST42, and the 3ST genome graph show values ranging from 0.9721 to 0.9973, 0.972 to 0.9967, 0.9738 to 0.998, and 0.9795 to 0.9999, respectively. The isolates of the ST154 group may be less diverse, as indicated by the smaller range of mapping rate variation, while the isolates of the ST41 group display greater diversity, as evidenced by the larger ranges of mapping rate variation (0.958 to 0.9956, 0.971 to 1, 0.9744 to 0.9988, and 0.9785 to 0.9998, respectively). The isolates of the ST42 group mapped to ST42 and the 3ST genome graph exhibit very similar mapping rates, but slightly lower rates when mapped to ST154 (0.9536 to 0.994) and ST41 (0.9559 to 0.9959). 
In summary, these findings suggest potential reference bias when using a single linear reference, and demonstrate that utilizing a graph as a reference can improve the mapping process.
\subsection{The clustering relationships among paths in the genome graph effectively reveal phylogenetic connections}
We evaluated the PGGB method's performance on a diverse group of 130 Neisseria meningitidis (NM) genomes, constructing a pangenome graph that proved more complex than the 4Sim and 3ST pangenomes. The 130NM pangenome graph spans 4,751,450 base pairs, over twice the size of single genomes, and comprises 629,349 nodes and 894,725 edges.
The 1D visualization of the 130NM graph, which colors paths based on orientation, shows genome chunks as either forward (black) or reverse (red) (Fig. 5A), illustrating the high recombination rate of Neisseria genomes. The 2D visualization using gafestus reveals a large bubble, potentially due to the substantial chunk of genomes aligned in reverse (Fig. 5B). This graph contains 135,368 SNPs (M), 26,126 indels, and 1,259 structural variations.
The all-vs-all alignment pangenome graph construction is unbiased, allowing distances among paths in the graph to effectively reveal genome relationships. Using the Jaccard similarity of the 130NM paths, we constructed a phylogenetic relationship among them. Clonal complexes are well-resolved by Jaccard similarity, with groups containing more than one genome clustering together (Fig. 5C). This finding is consistent with phylogenetic relationships unveiled by the kmer SNP-based analysis (Fig. 5D). Both the kmer SNP tree and the Jaccard similarity tree identify highly supported clades, but the branches in the kmer SNP-based analysis are more diverse. Overall, the all-vs-all alignment pangenome graph is suitable for relatively large genomes, capturing all types of variation and offering an unbiased method for genome comparison.
% For Original Research articles, please note that the Material and Methods section can be placed in any of the following ways: before Results, before Discussion or after Discussion.