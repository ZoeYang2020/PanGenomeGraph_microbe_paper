%% Materials & Methods

\subsection{Background of Neisseria meningitidis NZMenB epidemic strain}
In this study, we employed both real and simulated genomic data of Neisseria meningitidis to assess the pangenome pipeline, from pangenome graph construction to variant calling. Neisseria meningitidis is the primary agent responsible for bacterial meningitis, causing isolated incidents, outbreaks, and epidemics worldwide. In New Zealand (NZ), from 1991 to 2007, an extended serogroup B epidemic occurred due to a single strain known as NZMenB (designated B:4:P1.7-2,4), identified by the PorA variant (P1.7-2), and it still accounts for around one-third of meningococcal disease cases in NZ[22-24]. Based on our unpublished whole-genome sequencing (WGS) data, we categorized NZmemB into three phylogenetic clades, namely, clade154, clade41 and clade42 based on the multilocus sequence typing (MLST) of seven housekeeping genes, ST154, ST41 and ST42[25]. The majority of the disease during the epidemic was caused by ST154 and ST42, which are both monophyletic clades, while ST41 is more diverse.
\subsection{Nanopore long-reads}
To analyze the whole-genome sequencing (WGS) dataset, the original reference NC_017518 of ST42 was used. To obtain complete reference genomes for NMI01191 of ST41 and NMI97348 of ST154, we conducted Nanopore long-read sequencing. The high molecular weight genomic DNA was extracted using the Gentra Puregene Yeast/Bact. Kit (QIAGEN) and purified with Agilent Magnetic Beads. We used 400 ng of high molecular weight genomic DNA to construct sequence libraries utilizing the SQK-RBK004 Rapid Barcoding kit (Oxford Nanopore Technologies). The libraries were sequenced on R9.4.1 MinION flow cells. We used Flye_v.2.8-1[26] for de novo assembly, and Illumina sequencing reads were employed to polish the assembly using Unicycler_v.0.4.8[27]. Consequently, we were able to obtain complete NZMenB genomes (3STs) comprising NMI01191 of ST41, NMI97348 of ST154, and NC_017518 of ST42. The 3ST genomes were aligned using Mauve[28].
\subsection{Simulate genomes for pangenome graph construction}
Mauve alignment demonstrated big inversions among the three STs genomes.  To evaluate the pangenome graph construction, we simulated three genomes from NC_017518 of ST42 by introducing either randomly generated SNPs or mutated according to the SNP differences of ST41 and ST154 relative to ST42. The simulation was followed by introducing 200 indels and two inversions using simuG[29]. We named the three simulated genomes as ST42sim, ST41sim, and ST154sim. ST42sim contains random 5k SNPs, 200 indels, and two inversions relative to ST42, while ST41sim contains 2892 SNPs, 200 indels, and two inversions compared to ST42, and ST154 contains 4283 SNPs, 200 indels, and two inversions compared to ST42. We combined the three simulated genomes with ST42, which we referred to as 4Sim genomes, and used them for further analysis.
\subsection{Download diverse Neisseria genomes from NCBI}
To expand our evaluation of pangenome graph construction to more diverse genomes, we downloaded 130 Neisseria genomes from the NCBI (Table S1). The 130 Neisseria genomes (NM) were comprised of 8, 20, 20, 62, 2, 13, and 5 of group A, B, C, W, X, Y, and ungrouped, respectively.
\subsection{Pangenome graph construction by PGGB }
We constructed pangenome graphs for the 4Sim genomes, 3STs of NZmenB, and 131NM genomes using PGGB[30]. To construct the pangenome graphs, we initially aligned the start of ST41 and ST154 with ST42 for the 3STs, and all 130 genomes were fixed to start with the dnaA gene using circlater[31].
There are three essential parameters for pggb pangenome graph construction, i.e., -n, number of genomes, -s, segment length that defines the seed length used in wfmash, and -p, the minimum pairwise identity between seeds. To estimate the pairwise distance for each dataset, we used mash triangle[32]. For constructing the pangenome graph, we set the parameters -s, -p, -n to 1000, 96, 4 for the 4Sim genomes, 2000, 95, 3 for the 3STs, and 10000, 95, 131 for the 130NM genomes, respectively.  In addition, odgi -S was used to generate statistics of the seqwish and smoothxg graph. -m was turned on to generate MultiQC report of graphs' statistics and visualizations. All runs were executed with 48 threads on Dell R840 server, Xeon Gold 6244 3.60GHz CPU, 64 cores, and 3TB RAM at ESR. 
\subsection{Vg deconstruct to call variants in the graphs}
TTo identify both small and large variants from the pangenome graph, we employed the vg toolkit[16] to deconstruct the variants into VCF files using the reference NC_017518 of ST42. We set the parameters -a to process all snarls, including nested ones, -e to consider traversals that correspond to paths in the graph, and -K to retain conflicted genotypes to obtain the full list of variations in the graphs.
As the ground truth of the simulated genomes, ST42Sim, ST41Sim, and ST154Sim compared to ST42, were known, we compared the variations in the 4Sim genome graph against the ground truth. Initially, we filtered for variations larger than 100pb, and then we utilized vcfallelicprimitives from vcflib1.0.0[33] to break apart complex variations, which are less than 100bp. The variations identified in the graph were cross validated with the ground truth.
\subsection{Simulated NGS dataset of Neisseria Meningitidis for pangenome graph based variant calling}
In addition to the comparative genomics analysis of the paths (genomes) based on the genome graphs, these graphs can also serve as a pangenome reference for NGS data analysis. To evaluate the genome graph-based pipeline for NGS data mapping and variant calling using the vg toolkit[16], we simulated 100X 2X150 paired NGS data with an error rate of 0.5% using wgsim from samtools[34].
We began with eight genomes, which included the 3ST genomes and the three simulated genomes, and two mutated genomes, ST41Mut and ST154Mut, based on the SNP difference of ST41 and ST154 to ST42. To generate a set of 40 genomes, we initially introduced 2000 SNPs for each of the eight genomes with five repeats, followed by two additional rounds of 2000 SNPs (40 genomes per round) using SimuG[29]. Consequently, we obtained 128 genomes distributed among eight groups, including ST42, ST42Sim, ST41, ST41Mut, ST41Sim, ST154, ST154Mut, and ST154Sim.
\subsection{Real NGS dataset of NZmenB for pangenome graph based variant calling}
To test the graph-based analysis for real NGS dataset, we mapped 149 NZmenB isolates to the 3ST pangenome graph (Table S2). The 149 isolates include 49 from clade154, 48 from clade41 and 52 from clade42.
\subsection{Map the simulated NGS data to graph using the vg toolkit}
To map the NGS data to genome graph using the vg toolkit, we initially converted graphs (4Sim and 3ST) into 256bp chunks using vg mod -X 256. We then employed vg index to generate the index for the graph. Subsequently, vg map was utilized to map the NGS data against the graph. We also used vg stats to check the mapping statistics. 
To compare the mapping rates for NGS dataset against linear references verse the graph, we also mapping the NGS data to each linear refences using Bowtie2_v.2.3.2[35]. All sequenced and aligned reads were further processed using the Picard-tools-v.2.10.10[36] to remove duplicated reads and assessed with Qualimap_v2.2.1[37].     
\subsection{Variant calling for NGS data against genome graph}
There are two approaches to call variants in pangenome graphs: genotyping known variants and novel variants calling. In this study, we utilized both methods to call variants for the 128 simulated NGS data against the 4Sim genome graph. We only use the novel variants calling for the 149 NZmenB real dataset. 
To genotype known variants in the graph, we employed vg pack to calculate the support reads for each gam file. We then utilized vg call to genotype the known variants for each sample based on the snarls file generated from the 4Sim genome graph.
To consider novel variants from the reads, we employed vg augment to augment each gam file. Subsequently, we indexed the augmented graph, calculated read support for all variants, and performed variant calling. High confidence variants were identified using the PASS information and genotype (GT=1/1) from the VCF file. To evaluate the performance of variant calling in the context of simulated genomes, we compared the high confidence variants identified against the 4Sim graph with the simulated SNP records. Since the ground truth for ST41 and ST154 is unknown, both sets of simulated NGS data were excluded from this analysis. 
\subsection{Distance matrices for cluster detection}
To analyze the cluster relationship among the 130 NM genomes, we utilized odgi paths to extract jaccard distance matrices among paths in the pangenome graph. We then employed Hierarchical Clustering to construct the phylogenetic relationship among the genomes. To assess the accuracy of the clustering relationship, we compared it to the one obtained by kmer-based SNP phylogenetic analysis.
For kmer-based SNP analysis, we used ska, a reference-free, contig-based analysis, to extract the SNPs derived from kmer length 31 that were present in 90% of the isolates[38]. Phylogenetic analyses were constructed from the kmer-based SNP alignment using maximum likelihood (ML) under the best-fit model by Bayesian Information Criterion with iqtree-v.2.0.6. The robustness of the clades was estimated with 2000 ultra-fast bootstrap replicates.

\paragraph{Level 4}
\subparagraph{Level 5}
